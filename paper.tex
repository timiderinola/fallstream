\documentclass[runningheads]{llncs}
\usepackage[T1]{fontenc}
\usepackage{graphicx}
\usepackage{booktabs}
\usepackage[misc]{ifsym}
\newcommand{\corr}{(\Letter)}
% N.B.: do not change anything above this line. If you require additional packages, please load them directly after this line.
\usepackage{mwe}
% N.B.: you may delete the preceding line. It is used to display an example image in this template.

\begin{document}

\title{Watch Your Step: Realistic Fall Detection in Real-world Streaming Scenarios}

% Really important to test on mobilised data

\titlerunning{Watch Your Step: Real-time Fall Prediction Using Time Series Techniques}
% If the full title of your paper is short enough to also fit in the running head, you can omit the abbreviated paper title here. You can check as follows: if you comment out the \titlerunning line, something will appear in the header of all odd-numbered pages of your PDF from page 3 onward. This something is either the full title (in which case all is well), or the error message "Title Suppressed Due to Excessive Length". If this error message appears, you're going to want to provide an abbreviated title within the \titlerunning command, because if you won't do it, Springer will do it for you.

%N.B.: Author information (both in the \author{} and \authorrunning{} command) should only be present in the Camera-Ready Version of your paper. The version that you initially submit for review, ought to be double-blind. So, when initially submitting your paper, use:
\author{}
% \author{Andr\'e Lauren Benjamin\inst{1} \and
% Calvin Cordozar Broadus Jr.\inst{2,3} \corr \and
% Antwan Andr\'e Patton\inst{1}\orcidID{0000-1111-2222-3333}}
% You may leave out the orcidID information, if you want to.
% Use \corr to indicate the corresponding author. Note the spacing around the \corr command. Only one author can be the corresponding author.

%N.B.: comment out the \authorrunning{} command for the double-blind version of your paper submitted for review. Later, if your paper is accepted, use the command for the Camera-Ready Version.
\authorrunning{A.L. Benjamin et al.}
% First names are abbreviated in the running head.
% If there is one author, write 'A.L. Benjamin'.
% If there are two authors, write 'A.L. Benjamin and C.C. Broadus Jr.'
% If there are more than two authors, '[...] et al.' is used.

\institute{}
% \institute{Fictional Southern University, Savannah GA 31404, USA \email{\{a.l.benjamin,a.a.patton\}@fsu.fake}
% \and
% Fictional West Coast University, Long Beach CA 90840, USA \email{ccb@fwcu.fake}
% \and
% Secondary European Affiliation, Tiergartenstr. 17, 69121 Heidelberg, Germany
% \email{lncs@springer.com}}

\maketitle         

\begin{abstract}
Falls pose a significant health risk, particularly for older adults and individuals with medical conditions. Current fall detection techniques are often limited by the use of simulated datasets or segmentation methods that require prior knowledge of fall occurrences. This paper introduces a novel real-time fall probability framework leveraging more than 24 hours of real-world mobility and fall data collected from 41 participants using inertial measurement units in the FARSEEING dataset. We employ state-of-the-art time series techniques to detect fall probabilities in streaming mode without prior knowledge of fall events, eliminating the need for manual feature engineering. We further validate our FARSEEING-trained models on a continuous 7-day dataset from a single subject, containing four real-world falls. Our approach demonstrates high computational efficiency and achieves F$_1$ score up to 0.89 and a false alarm rate as low as 0.085 per hour on FARSEEING. This framework demonstrates the potential for real-world implementation, offering a promising step toward more realistic and effective fall management strategies, potentially reducing fall-related injuries and improving the quality of life for at-risk individuals.

150--250 words.

\keywords{Fall prediction  \and Time series \and Real-world falls.}
\end{abstract}

\section{Introduction}
Falls constitute a major global health concern, representing the second leading cause of unintentional injury deaths worldwide, claiming an estimated 684,000 lives annually \cite{step_safely_2021}. People living with certain medical conditions and older adults, particularly those over 60, are at the highest risk \cite{vaishya2020falls}. Beyond fatalities, an estimated 37.3 million falls annually require medical attention, placing a significant burden on healthcare systems \cite{camp2024integrating}. Therefore, rapid fall detection is crucial for mitigating the severity of fall-related injuries and facilitating timely interventions. Given this critical need, researchers have explored various approaches to automatically identify fall events.

A fall is an event that results in a person coming to rest unintentionally on the ground, floor, or other lower level \cite{step_safely_2021}. Automatically detecting a fall involves data capture, preprocessing, feature extraction, and classification \cite{liu2023review}. Since falls are unintentional by nature, the first and most important step of fall detection, which is data capture, is challenging. This has resulted in the widespread use of simulated fall data for training fall detection models. However, models trained on simulated falls have been shown to exhibit greatly degraded performance in real-world scenarios \cite{aderinola2024accurate}.

Fall data capture involves recording the daily activities of participants for a set period of time in order to capture the characteristic features of their normal activities of daily living (ADLs) and falls. Such data can be recorded using wearable devices such as inertial measurement units (IMUs), or environmental devices, such as cameras and ambient sensors \cite{gaya2024deep}. However, due to their low-cost, portability, and efficiency, wearable devices are often the capture devices of choice, especially for long-term data capture free-living environments \cite{mohan2024artificial}.

In order to distinguish between falls and ADLs, algorithms use could be threshold-based or machine learning (ML) based. Threshold-based methods such as \cite{de2021wearable} use cut-off values set on the sensor signals commonly have high false alarm rates, which could lead to "false alarm fatigue" \cite{mosquera2020automated}. On the other hand, ML methods such as  use conventional classifiers with manually crafted features \cite{son2022machine}, or deep representation learning \cite{liu2023deep}. Some more recent methods take a hybrid approach of preprocessing signals with set thresholds before passing them to an ML model \cite{fernandez2024edge}. However, most of these methods involve segmentation techniques that require prior knowledge of the occurrence of the fall in the test data, limiting their real-world applicability. Furthermore, developing robust fall detection systems for real-world applications presents unique challenges, including the diversity of fall characteristics and the need for continuous, real-time monitoring.

Despite these advances in fall detection, a major limitation still remains the limited applicability in real-world scenarios, partly due to simulated data and partly due to unrealistic segmentation. Existing methods often struggle to operate effectively in truly real-time, continuous monitoring scenarios. This paper addresses this gap by introducing a novel real-time fall probability framework that operates on continuous sensor data without requiring prior knowledge of fall events. Our main contributions are:

\begin{enumerate}
    \item We present a framework for fall probabilities in streaming scenarios, which can improve fall detection outcomes by reducing false alarms and enabling the potential identification of pre-fall events.
    \item We demonstrate realistic fall detection with fall probabilities on two real-world datasets.
\end{enumerate}

To satisfy conditions ideal for a real-time streaming environment, we perform no segmentation on the test set. Additionally, we perform no feature engineering and use efficient time series techniques. Our approach is evaluated on the FARSEEING dataset \cite{klenk2016farseeing}, a large real-world dataset with 92 fallers (mean age 76.1$\pm$12.6 years) and 208 verified falls captured using inertial sensors (accelerometer data). We also further evaluate models trained on FARSEEING on a continuous 7-day real-world accelerometer dataset of a subject, containing 4 falls. We use a fixed-size overlapping window approach for training. On the test set, we take a streaming approach by sliding over the full signal for each test participant and obtaining the probability of each window being a fall event.


\subsection{Main Contributions}
Please note that the first paragraph of a section or subsection is
not indented. The first paragraph that follows a table, figure,
equation etc. does not need an indent, either.

Subsequent paragraphs, however, are indented.

\section{Related Work}
In this section, we discuss recent approaches to fall intervention, namely, reactive approaches, and proactive approaches. While reactive approaches focus on detecting falls as soon as possible after it happens to enable quick response and management, proactive approaches focus on taking measures to prevent falls altogether.
\subsection{Reactive Approaches}
\subsection{Proactive Approaches}
% \subsection{Basket Weaving}
% \subsubsection{Underwater Basket Weaving} Only two levels of
% headings should be numbered. Lower level headings remain unnumbered;
% they are formatted as run-in headings.

% \paragraph{Underwater Basket Weaving Under Difficult Circumstances}
% The contribution should contain no more than four levels of
% headings. Table~\ref{tab1} gives a summary of all heading levels.

\begin{table}[t]
\caption{Table captions should be placed above the
tables.}\label{tab1}
\begin{tabular}{lll}
\toprule
Heading level &  Example & Font size and style\\
\midrule
Title (centered) &  {\Large\bfseries Lecture Notes} & 14 point, bold\\
1st-level heading &  {\large\bfseries 1 Introduction} & 12 point, bold\\
2nd-level heading & {\bfseries 2.1 Printing Area} & 10 point, bold\\
3rd-level heading & {\bfseries Run-in Heading in Bold.} Text follows & 10 point, bold\\
4th-level heading & {\itshape Lowest Level Heading.} Text follows & 10 point, italic\\
\bottomrule
\end{tabular}
\end{table}

\section{Materials and Methods}

Displayed equations are centered and set on a separate
line.
\begin{equation}
x + y = z
\end{equation}
Please try to avoid rasterized images for line-art diagrams and
schemas. Whenever possible, use vector graphics instead (see
Fig.\@ \ref{fig1}).

\begin{figure}[t]
\includegraphics[width=\textwidth]{example-image-duck}
\caption{A figure caption is always placed below the illustration.
Please note that short captions are centered, while long ones are
justified by the macro package automatically.} \label{fig1}
\end{figure}

%
% the environments 'definition', 'lemma', 'proposition', 'corollary',
% 'remark', and 'example' are defined in the LLNCS documentclass as well.
%
\begin{proof}
Proofs, examples, and remarks have the initial word in italics,
while the following text appears in normal font.
\end{proof}
For citations of references, we prefer the use of square brackets
and consecutive numbers. Citations using labels or the author/year
convention are also acceptable. The following bibliography provides
a sample reference list with entries for journal
articles~\cite{ref_article1}, an LNCS chapter~\cite{ref_lncs1}, a
book~\cite{ref_book1}, proceedings without editors~\cite{ref_proc1},
and a homepage~\cite{ref_url1}. Multiple citations are grouped
\cite{ref_article1,ref_lncs1,ref_book1},
\cite{ref_article1,ref_book1,ref_proc1,ref_url1}.

\section{Experiments}
% POST-PROCESS CLASSIFIER OUTPUT USING EXPERT KNOWLEDGE
    % it is acceptable to have a tolerance of 10 seconds before and after the window 
    % Use cohort-based splitting to avoid having subjects in both train and testing
    % Have some rules about how close detected falls can be
% 1. Discuss data
% 2. Preprocessing, segmentation, etc
% 3. Fall prediction--how far back can we go before the classifier begins to fail completely?
% First step, leave test data as before, augment training with ADLs
% 4. Detecting when we enter that window.
% 5. Effects of personalisation
% 5. Testing on another dataset--mobiliseD
\subsection{Experimental Setup}
\subsection{Experimental Results}

\section{Discussion}

\section{Conclusion}

Of course, authors have complete freedom on how they choose to structure their paper. Section headers from Introduction up to and including Conclusions are completely up to the discretion of the authors; use whichever structure you see fit. Title, Abstract, the credits environment, and References, however, are mandatory.

\begin{credits}
\subsubsection{\ackname} A bold run-in heading in small font size at the end of the paper is
used for general acknowledgments, for example: This study was funded
by X (grant number Y).

\subsubsection{\discintname}
It is now necessary to declare any competing interests or to specifically
state that the authors have no competing interests. Please place the
statement with a bold run-in heading in small font size beneath the
(optional) acknowledgments,
for example: The authors have no competing interests to declare that are
relevant to the content of this article. Or: Author A has received research
grants from Company W. Author B has received a speaker honorarium from
Company X and owns stock in Company Y. Author C is a member of committee Z.
\end{credits}
%
% ---- Bibliography ----
%
% BibTeX users should specify bibliography style 'splncs04'.
% References will then be sorted and formatted in the correct style.
%
\bibliographystyle{splncs04}
\bibliography{refs}
%% Note that this preceding line implies that you store your BibTeX references in a file called 'mybibliography.bib'. If you instead store your references in a file with a different name, for instance 'references.bib', the preceding line should read '\bibliography{references}'. Whatever you do, DO NOT put the file name extension .bib inside the \bibliography command; this will trip up LaTeX compilers. 
%
% If you do not want to use BibTeX, you can also type up the bibliography exactly as you see fit, using the following structure:
% \begin{thebibliography}{8}
% % Note that this number 8 reserves an amount of space (equal to the natural width of the given number) for the label of your references; if you have more than 9 references, you will want to change this number to 18. If you have more than 19 references, this number is best changed to 88. If you have more than 99 references, I salute you.
% \bibitem{ref_article1}
% Author, F.: Article title. Journal \textbf{2}(5), 99--110 (2016)

% \bibitem{ref_lncs1}
% Author, F., Author, S.: Title of a proceedings paper. In: Editor,
% F., Editor, S. (eds.) CONFERENCE 2016, LNCS, vol. 9999, pp. 1--13.
% Springer, Heidelberg (2016). \doi{10.10007/1234567890}

% \bibitem{ref_book1}
% Author, F., Author, S., Author, T.: Book title. 2nd edn. Publisher,
% Location (1999)

% \bibitem{ref_proc1}
% Author, A.-B.: Contribution title. In: 9th International Proceedings
% on Proceedings, pp. 1--2. Publisher, Location (2010)

% \bibitem{ref_url1}
% LNCS Homepage, \url{http://www.springer.com/lncs}, last accessed 2023/10/25
% \end{thebibliography}
\end{document}
